%%%%%%%%%%%%%%%%%%%%%%% file typeinst.tex %%%%%%%%%%%%%%%%%%%%%%%%%
%
% This is the LaTeX source for the instructions to authors using
% the LaTeX document class 'llncs.cls' for contributions to
% the Lecture Notes in Computer Sciences series.
% http://www.springer.com/lncs       Springer Heidelberg 2006/05/04
%
% It may be used as a template for your own input - copy it
% to a new file with a new name and use it as the basis
% for your article.
%
% NB: the document class 'llncs' has its own and detailed documentation, see
% ftp://ftp.springer.de/data/pubftp/pub/tex/latex/llncs/latex2e/llncsdoc.pdf
%
%%%%%%%%%%%%%%%%%%%%%%%%%%%%%%%%%%%%%%%%%%%%%%%%%%%%%%%%%%%%%%%%%%%


\documentclass[runningheads,a4paper]{llncs}
\usepackage[utf8]{inputenc}
\usepackage{amssymb}
\setcounter{tocdepth}{3}
\usepackage{graphicx}

\usepackage{url}

\newcommand{\keywords}[1]{\par\addvspace\baselineskip
\noindent\keywordname\enspace\ignorespaces#1}

\usepackage{algorithmic}
\usepackage{algorithm}
\usepackage{mathtools,xparse}


% your email address goes here
\urldef{\mailsa}\path|your-emails@suffix| 
\newcommand{\ud}{\,\mathrm{d}}
\newcommand{\ngbf}[1]{\mathbf{#1}}
\newcommand{\gbf}[1]{\boldsymbol{#1}}
\newcommand{\E}{\mathrm{E}}
\newcommand{\Cov}{\operatorname{Cov}}
\newcommand{\diag}{\operatorname{diag}}
\newcommand{\EI}{\operatorname{EI}}
\newcommand{\PI}{\operatorname{PI}}
\renewcommand{\P}{\operatorname{P}}

\DeclarePairedDelimiter{\norm}{\lVert}{\rVert}
\newcommand{\mcite}[1]{[Fehlendes Zitat: #1]}
\newcommand{\hvs}{SMS-EMOA}
%------------------------------------------------------------------------75
\newcommand{\xx}{{\bf x}}
\newcommand{\XX}{\mathbf{X}}
\newcommand{\RBB}{\mathbb{R}}
\newcommand{\ndim}{d}
\newcommand{\ybf}{\mathbf{y}}
\newcommand{\SBB}{\mathbb{S}}
\newcommand{\shat}{\hat{s}}
\newcommand{\FCAL}{\mathcal{F}}
\newcommand{\fine}{\hspace*{\fill}$\Box$}

\graphicspath{{img/}}

%\parskip 2pt 
%\raggedbottom
\usepackage{listings}
\usepackage{color}
\definecolor{lightgray}{rgb}{.9,.9,.9}
\definecolor{darkgray}{rgb}{.4,.4,.4}
\definecolor{purple}{rgb}{0.65, 0.12, 0.82}

\lstdefinelanguage{JavaScript}{
  keywords={typeof, new, true, false, catch, function, return, null, catch, switch, var, if, in, while, do, else, case, break},
  keywordstyle=\color{blue}\bfseries,
  ndkeywords={class, export, boolean, throw, implements, import, this},
  ndkeywordstyle=\color{darkgray}\bfseries,
  identifierstyle=\color{black},
  sensitive=false,
  comment=[l]{//},
  morecomment=[s]{/*}{*/},
  commentstyle=\color{purple}\ttfamily,
  stringstyle=\color{red}\ttfamily,
  morestring=[b]',
  morestring=[b]"
}

\lstset{
   language=JavaScript,
   backgroundcolor=\color{lightgray},
   extendedchars=true,
   basicstyle=\footnotesize\ttfamily,
   showstringspaces=false,
   showspaces=false,
   numbers=left,
   numberstyle=\footnotesize,
   numbersep=9pt,
   tabsize=2,
   breaklines=true,
   showtabs=false,
   captionpos=b
}

\begin{document}

\mainmatter  % start of an individual contribution

% first the title is needed
\title{Applying ACO to tower defense routing}

% a short form should be given in case it is too long for the running head
\titlerunning{ACO on Tower Defense}

% the name(s) of the author(s) follow(s) next
%
% NB: Chinese authors should write their first names(s) in front of
% their surnames. This ensures that the names appear correctly in
% the running heads and the author index.
%
\author{Matthias Müller-Brockhausen \and Mark van Koningsveld}
%
\authorrunning{Matthias Müller-Brockhausen \and Mark van Koningsveld}
% (feature abused for this document to repeat the title also on left hand pages)

% the affiliations are given next; don't give your e-mail address
% unless you accept that it will be published
\institute{Leiden Institute of Advanced Computer Science\\
Leiden University, Niels Bohrweg 1, 2333 CA Leiden, The Netherlands}

%
% NB: a more complex sample for affiliations and the mapping to the
% corresponding authors can be found in the file "llncs.dem"
% (search for the string "\mainmatter" where a contribution starts).
% "llncs.dem" accompanies the document class "llncs.cls".
%

\toctitle{Lecture Notes in Computer Science}
\tocauthor{Authors' Instructions}
\maketitle


\begin{abstract}
In this paper we describe two experiments. The first experiment is to see if it is possible to use Ant Colony Optimization to route creeps within a Tower Defense Game. We first see if they find the shortest route and after that the one with the least risk of death. We do this with three levels with different characteristics. In the second experiment we compare multiple settings for the algorithm and try related algorithms to see which keeps the most creeps alive and is the fastest. We evaluate which strategy performs best under which parameters and conclude that some of the strategies are very well suited for the task.
\end{abstract}




\section{Introduction}
\label{sec:introduction}
In this paper we introduce the topic of ant colony optimisation and tower defense games as well as related work in the field. After that we think up our own tower defense game, described in section \ref{sec:implementation}, and try to apply Ant Colony Optimization Algorithms explained in the following sections to it in order to route the creeps through the map.

\subsection{Ant colony foraging behaviors}


Social insects like bees and ants that are not sophisticated by themselves can show complex behaviors when working as a colony. Ants organise themselves using pheromones. These are left by every single ant creating trails that connect food sources. The organisation of these trails are an emergent property of accumulating pheromones\cite{blum2005ant}.

While exploring their surroundings there will be a bigger chance that he will choose the road that has more pheromones on it creating a positive feedback loop outside of the ants individual control, which is called stigmergy. These trails can stay for a long time in the same place while slowly evaporating\cite{cordon2002review}.

In the double bridge experiment by Deneubourg et. al. these features were used to find the shortest route on two bridges. Because the ants that chose the shorter route arrived at the destination earlier and had left pheromones they also tended to take the same route back, further reinforcing the trail. When the second ant arrived they had more chance to also take the short route back because of the two layers of pheromones already present creating more feedback\cite{deneubourg1990self}. Even when there is no shorter route the ants will eventually choose one of the two sides because of stochastic oscillations. When there is a pheromone trail established it will remain the preferred route even if a shorter one is introduced because the chance of enough ants taking the other route to shift the balance is very low\cite{dorigo2006introduction}.

\subsection{Simple Ant Colony Optimization for connected graphs}

The abilities of ant colonies can be simulated to find the shortest route in graph based problems. These artificial ants are adapted to function in the virtual world and to evade pitfalls in complicated graphs \cite{gambardella1999macs}.
Single artificial ants that act on their own are sent down the different paths making stochastic decisions at every node and construct solutions. The quality is measured by how much time was needed or nodes visited on a group level.\cite{dorigo2010ant} Artificial ants have a forward mode and a backward mode. During forward mode the ant makes probabilistic choices to get to the end of the graph which are stored in the memory of the ant. In backward mode loops are deleted from the trail by revisiting the nodes and looking for doubles.Only if the trail is shorter than existing trails there are pheromones added to the visited nodes. In addition this stops the ants from getting stuck in their own pheromone feedback loop \cite{dorigo2001experimental}.

To avoid local optimums the concept of pheromone evaporation is used so that the ants do not get stuck in a sub optimal route. \cite{korb2007ant} These ant colony based algorithms that solve shortest path problems and problems like the TSP are versions simple ant colony optimization techniques or S-ACO, the most basic one shown below\cite{dorigo1997ant}.  Multiple versions of this algorithm exist like the ANT system, MAX/MIN ant system and the ACS system. These differ from the ant colony metaheuristic that can be used to solve any optimization problem \cite{stutzle1999max}.

\subsection {Application of S-ACO’s to tower defense games}

Artificial Intelligence has been used in video games to make the game more challenging or more interactive. It has allowed enemies to be more adaptive and partners to be more interactive. Furthermore it can be used to make stories adaptive in for instance adventure games, let enemies change strategies or to have them interact with the environment in a natural way\cite{laird2001human}.  Tower defense games are a type of strategy game that involves a road on which enemies traverse and a human player that has to build towers along the way. The enemies or "creeps" come in waves and try to reach the base of the player at the end of the path. The player gets points and resources for every enemy destroyed but when too much creeps reach the end the player loses. Most of the time there are creeps with varying speeds and armor ratings. Some tower defense games only have a single road to traverse while others have connected roads with multiple options for routing \cite{tan2013automated}. In the rest of this paper we will refer to the \textit{ants} as \textit{creeps} because it fits the tower defense setting.

AI can be used in different ways to influence the gameplay of tower defense games. It can be used for map generation using randomly placed nodes that are connected by for instance a shortest path algorithm. \cite{avery2011computational} In research by \cite{huo2009application}  both Particle Swarm Optimization and Genetic Algorithm metaheuristics are used to test a map by placing towers in the best places\cite{huo2009application}.  In a different research the algorithm NSGA-II was used to vary the amount of creeps that are sent to react to the players abilities\cite{ariyadicreep}. 

In the paper by Guimarans\cite{guimarans2016multi} not only lengths of a connected graph are given but also the risk factors of each path are used as an extra dimension. In this way there was a path calculated that is the shortest but has the highest risk, and a path that is longer but has less dangers along the way. This was used to calculate if a road that is shorter but has traffic problems would require less travel time than a road that was longer but where there are no traffic problems\cite{guimarans2016multi}.  This two dimensional approach could be used to both find the shortest route through a tower defense game and account for the towers along the way as a risk factor. 






\subsection{Basic Math behind ACO}

In ACO the most important move for us is in the tower defense game is the \textit{State Transition Rule}. This is the formula that a creep uses to decide where it goes to next as displayed here\cite{dorigo2006ant}:

\begin{equation}
p^k_{ij}=
\begin{cases}
\frac{\tau^\alpha_{ij}*n^\beta_{ij}}{\sum_{ciL\in N(s^P)}\tau^\alpha_{il}*n^\beta_{il}} \\
0 otherwise
\end{cases}
\end{equation}

Looking at the formula in depth it is the probability that creep k, standing on node i will move to city j\cite{maniezzo2002ant}.
There are two factors that have an influence this probability.
First we have the pheromone level on the trail called $\tau^\alpha_{i\psi}$ which may optionally be weighted using $\alpha$.
Then there may also be an attractiveness parameter that uses a priori knowledge to influence the result called $n^\beta_{i\psi}$ which can optionally be weighted using $\beta$\cite{dorigo2006ant}.

Put into simple words this says the higher the pheromone level the higher the probability a path is chosen, but in the end it is still random which path is chosen, it is merely a weight for the choice itself.
The trail pheromone level $\tau^\alpha_{i\psi}$, will in our case be influenced based on the creeps chosen behavior explained in section \ref{sec:behavior}. This chosen behavior also influences the weights and may introduce slight alterations to the formula above.

\section{Implementation}
\label{sec:implementation}
This chapter is split up into two subparts: Tech-Stack and Actual Implementation. The Tech-Stack findable in Section \ref{sec:implementationtech}, explains the libraries and techonolgies that enable our implementation. The Actual Implementation in Section \ref{sec:implementationactual} covers how we ensured a minimally working tower-defense game.

\subsection{Technical Stack}
\label{sec:implementationtech}
For our expirment to work, we require a programatically executable and influencable version of a tower defense game. There are definitely many open-source solutions out there\footnote{A search on github for "tower defense" in javascript returned 380 results.\cite{githubtowerdef}}, but for our ant colony optimization applicability we only require a very basic and simple implementation, so we decided to quickly roll out our own. How this actual Implementation looks is laid out in Section \ref{sec:implementationactual}

For the implementation we chose to create a web-tool through the usage of Javascript as a programming langauge. It provides the major advantage that, if we want to let someone else run our program to verify our conclusions, they can simply visit a link in their browser, be it mobile, desktop or gaming-console, to interact with our tool and visually see ant's deciding their ways and the pheromone-levels on the street slowly increasing\cite{curran2012future}.

We will be using typescript\cite{libstypescript} as an intermediate language to Javacsript because it supports react's JSX-Syntax per default so no need to use a transpiler like babel, and having type-suggestions etc. increases developer workflow speed as well as eases understanding of code for other team-members.

In order to make sure changes do not break any existing behaviour we use the test runner called \textit{jest}\cite{libsjest}. It comes with the handy utility  \textit{matchSnapshot}\footnote{Example Usage see Section \ref{src:testgamesim}}, which allows us to match the whole current gamestate against a snapshot and immediately see unwanted changes if they occur. This \textit{matchSnapshot}-Function is also capable of capturing a snapshot of our entire UI-render, which means we are able to achieve 100\%-Line-Code-Coverage

In order to be able to easily implement the UI and add Animations merely based on the state of the game we chose to opt for \textit{react}\cite{libsreact}, with the easy \textit{velocity-react}\cite{libsreactvelocity} animation extension. With this we only have to write the position that we want for example an ant to be displayed at for the current tick, and \textit{velocity-react} remembers the previous values and makes sure that the changes in position values are then animated. This creates the illusion of the ants walking along the path as the ticks progress\footnote{Actual Code see Section \ref{src:uiant}}. This also inherently comes with some weird animation behaviour, for example when setting the tickspeed to high then the ants position animation is merely from start to finish so it may appear to not walk on a valid street while in the backend it definitely did. Also with each change in the tick there is a slight "freeze" in the movement of the ants.

\subsection{Actual Implementation}
\label{sec:implementationactual}

In order to Implement a tower defence game one has to identify the key-objects that play different roles in the game namely that is
\begin{itemize}
\item The walkable path with special start and target point
\item Ants walking along this path
\item Towers trying to damage these Ants
\end{itemize}
First of we store our whole map-state as well as the map-layout itself in a single class called \textit{GameField}\footnote{Source-Code in Section \ref{src:logicgame}}. The game should hold a simple 2 dimensional coordinate system, that has an x and y (width and height) constraint, so we can place our key-objects within the Field. Towers\footnote{Source-Code in Section \ref{src:logictower}} and Street\footnote{Source-Code in Section \ref{src:logicstreet}}-Parts both originate from the class \textit{Tile}\footnote{Source-Code in Section \ref{src:logictile}} which lets us position them within the coordinate system of the \textit{GameField} and makes sure they cannot overlap.
The Streets are a simple adjacency List meaning each street-part only nows its previous and next nodes and nothing else. This makes our street-network a simple network suitable for ACO. Streets also have a list of ants that are currently on them and the current numerical pheromone level we introduced in Section \ref{sec:introduction}. This is already all the data needed for the \textit{Ant}\footnote{Source-Code in Section \ref{src:logicant}}-Class to make it's ACO-Based Routing decisions!
The Tower is also very simple it has a range, damage amount per hit and amount of hits per tick. Each Tick it will look for all streets in range and then attack as many ants as possible. It will try to damage ants that are the furthest and have the fewest HP first.

Bringing all this domain-logic to a visually interpretable is also pretty simple with the given data-structure. For each Tile we simply display a basic square via an HTML-\textit{div}-Element. Since we have a certain x and y-limit for our GameField mentioned earlier, we can calculate the exact width, height and position of each tile width within our window-JS-Object\cite[P. 570]{goodman2002dynamic}.

Each Game Instance also holds a variable that determines what algorithm the Ant's use to determine which node to walk to next, and how strongly pheromones are influenced and how they decay. Through that we will be able to run different kinds of experiments on different maps.

The Game itself is Tick-Based. In each tick we first let the tower's deal all their damage. Then all the ants decide where they walk to next. After that has happened the general decay of pheromones is processed if wanted.

This is already all the implementation. To sum it all up: We have an isolated GameField-Class that can instantiate a whole test-setup including map, ants and tick-progression, so in theory it is possible to use this code to run multiple expirements at the same time.


\section{Experiments}
\label{sec:experiments}

In this Section we will cover what kind of experiments we run as well as which data we collect and evaluate for these.

\subsection{Setup}

Because our Implementation can run headless we can collect much of the wanted data described in Section \ref{sec:experimentevdata} automatically. 

We chose the following Map-Setups 

\subsection{Evaluated Data}
\label{sec:experimentevdata}

In order to evaluate how well the experiment worked, we need to think about what the actual target of our algorithm is. Revisiting the Introduction in Section \ref{sec:introduction}, reminds us that the target is that the ants find an ideal way to our target.

Since it has been proven that the distribution quickly converges\cite[P. 15]{maniezzo2002ant.pdf}, which is quite logical since stronger reinforced paths are more likely to attract new ants and get reinforced even more, we can say that one of our targets is that the pheromone-level on specific Street-Segments has reached at least a value of x.
Because we have a tick-based implementation we know the amount of ticks needed until the target pheromone value was reached. Additionally we can also involve the percentage of dead ants into the evaluation because on some maps it gives us an indication about how strongly the ants have wandered  the wrong hurtful paths. It is important to use the percentage of dead ants here, since the amount of ants spawned per tick is random for each tick!


\subsection{Ant Behaviour Algorithms}
We can set various behaviours / target searching algorithms for the ants.
The following sums them up

\section{Results}
\label{sec:results}
This section is divided into two subsections. First we present the results of manual experiments that we obtained by hand via the UI that we implemented in section \ref{sec:manualver}.
The second section \ref{sec:autover} contains results we obtained and evaluated automatically by running the simulations headlessly.

\subsection{Manual Verifications}
\label{sec:manualver}
insert here text we wrote on google docs on meeting day


\subsection{Automatic Evaluation Results}
\label{sec:autover}
For the automatic evaluation we tried varying settings and ran them across all maps. Because the amount of different parameters and the hence large amount of graphs to interpret we will here only mention and display the settings that we deemed relevant information. To be statistically relevant every experiment setting is run 50 times and we then take the mean value of that for the graph. We also always include Towers in the maps.

Regarding the settings when we talk about normal settings, then these are the settings we found to work best using the manual mode. This consist of a spawn-amount of 14, a decay-threshold of -0.007 and a pheromone-increase threshold of 0.01.

First of we will start out by excluding the continuous vapor in the future Figures because as Figure \ref{fig:mirrorvaptick} clearly shows, the runtime is much larger than for all other three behaviors for the mirrored map. This behavior was visible across all maps and settings, another example that stresses the uselessness of the continuous vapor is the runtime graph portrayed in Figure \ref{fig:shortlongvaptick} for the short and long map, the very basic map that all of them should be able to easily master.

\begin{figure}[H]
  \centering
  \includegraphics[width=1\linewidth]{images/normalmirroredwithtower-ticks-line}
  \caption{Comparing required Tick-Amount on the y-Axis to the target amount of creeps to reach the target on the x-Axis for the mirrored map with normal settings.}
  \label{fig:mirrorvaptick}
\end{figure}

\begin{figure}[H]
  \centering
  \includegraphics[width=1\linewidth]{images/normalshortandlongwithtowers-ticks-line}
  \caption{Comparing required Tick-Amount on the y-Axis to the target amount of creeps to reach the target on the x-Axis for the short and long map with normal settings}
  \label{fig:shortlongvaptick}
\end{figure}

With the automatic evaluation we also further prove, that deducing pheromones upon death is rather counter-productive than helpful as visible in Figure \ref{fig:deathsubshitty}. But it does make sense given that any street-part an ant crossed on it's way to death will be deduced in pheromone. That means even a correct path like in the short and long map the longer one can lead to pheromone deduction and hence make the path unattractive again and future ants more likely to walk the even more dangerous path with more deaths.
Also for the weighted shortest path leaving death-substraction out even slightly improves the result. As for the non continuous vapor behavior, not only does substraction not matter as much, but also all three perform quite similar.


\begin{figure}[H]
  \centering
  \includegraphics[width=1\linewidth]{images/normalshortandlongwithtowers-deaths}
  \caption{Comparing deaths for the different creep behaviors with and without substracting pheromones upon death on the short and long map with normal settings}
  \label{fig:deathsubshitty}
\end{figure}

The following graph, depicted in Figure \ref{fig:threesame} which doesn't include the continuous vapor any more shows, shows that the remaining three behaviors perform similar not only regarding death amount but also regarding their runtime.

\begin{figure}[H]
  \centering
  \includegraphics[width=1\linewidth]{images/normalsquaremaze-ticks-line}
  \caption{Comparing required Tick-Amount on the y-Axis to the target amount of creeps to reach the target on the x-Axis for the maze map with normal settings}
  \label{fig:threesame}
\end{figure}

Because of that similarity we ran another experiment only using the shortest weighted path algorithm, but with a few differing settings. The result of that is accessible through Figure \ref{fig:diffsettings} which portrays the amount of Ticks and Figure \ref{fig:diffsettingsdeath} which displays the amount of deaths.

As we can see using a high spawn rate with no or low decay will result in much more deaths than usual. Manually reconstructing this scenario with the UI explains why that is: With a high spawn rate many ants come through every way regardless of the tower strength. Hence bad ways can be reinforced and made attractive even though they are not.
Another possible conclusion is that using a low spawn rate in combination with no or low decay will increase runtime, which is logical given that the target is the amount of ants that need to reach the target and less spawned ants mean less ants per tick are able to arrive at the target. 

\begin{figure}[H]
  \centering
  \includegraphics[width=1\linewidth]{images/mirroredwithtower-ticks-line}
  \caption{Comparing required Tick-Amount on the y-Axis to the target amount of creeps to reach the target on the x-Axis for the mirrored map with normal settings}
  \label{fig:diffsettings}
\end{figure}

\begin{figure}[H]
  \centering
  \includegraphics[width=1\linewidth]{images/mirroredwithtower-deaths-line}
  \caption{Comparing amount of deaths on the y-Axis to the target amount of creeps to reach the target on the x-Axis for the mirrored map with normal settings}
  \label{fig:diffsettingsdeath}
\end{figure}

We have also already shown that the spawn-amount only has a logical impact on the tick amount and 14 is a good number given the damage of the towers. Now we also want to prove that the pheromone increase and decrease-level we declared as \textit{normal} are optimal
For that Figure \ref{fig:diffsettings2} compares the amount of ticks and Figure \ref{fig:diffsetting2sdeath} the amount of deaths for different pheromone-settings with the spawn amount 14 on the mirrored map.

As we can see the best value is in both instances achieved by the completely normal settings. It also shows that a low increase or also decay can negatively influence runtime. Also using a low pheromone increase can cause more deaths.

\begin{figure}[H]
  \centering
  \includegraphics[width=1\linewidth]{images/mirrorednormalpheromticks}
  \caption{Comparing required Tick-Amount o n the y-Axis to the target amount of creeps to reach the target on the x-Axis for the mirrored map with varying settings}
  \label{fig:diffsettings2}
\end{figure}

\begin{figure}[H]
  \centering
  \includegraphics[width=1\linewidth]{images/mirroredwittowerdeaths}
  \caption{Comparing amount of deaths on the y-Axis to the target amount of creeps to reach the target on the x-Axis for the mirrored map with normal settings}
  \label{fig:diffsetting2sdeath}
\end{figure}
\section{Future work}
\label{sec:futurework}
Our experiments are far from exhaustive, but limited by time. Hence it is also important to point out what things still remain to be explored in future work.

The tower defense implementation described in section \ref{sec:implementationtech} is pretty basic and more a simulation than a real world one.
In a real tower defense game setting the towers will be built over time.  Furthermore the amount of towers that the player can build depends on the points that are gathered through killing creeps.
Usually one also distinguishes between different types of towers that deal different amounts of damage can use other target methods (area vs. single target damage)\cite[P. 57]{hernandez2015mathematics}.

Another important difference is that the spawend creeps in a regular tower defense game vary. This variation can encompass things like movement speed, health and even damage to towers\cite[P. 57]{hernandez2015mathematics}. Each of these is a potential parameter that could influence and be taken into account for the routing.

Another possible improvement would be to analyse even more maps than the one we described in section \ref{sec:testmaps}.



\section{Conclusion}
\label{sec:conclusion}
Ant Colony Optimization is \cite[P. xy]{dorigo2006ant}


\section{Appendix}
\label{sec:appendix}
Appended are only Files that were referenced in the description in section \ref{sec:implementationactual}.

\section{Source Code for GameField-Class}
\label{src:logicgame}

\begin{lstlisting}
import { filter, find, isNil, remove, union } from "lodash"
import noOp from "../noOp"
import randomNum from "../randomNum"
import Ant from "./ant"
import Street from "./street"
import Tile, { tileTypes } from "./tile"
import Tower from "./tower"

const experimentChoices = {
  onlyOnSuccess: 6,
  /**
   * Rank-based ant system (ASrank) = shortest path
   */
  shortestPathWeight: 8,
  shortestPathOnly: 9,
  /**
   * also called Ant colony system Algorithm
   */
  continousVapor: 11
}

class GameField {
  public timerId: any
  public fieldSizeX: number
  public fieldSizeY: number
  public antList: Ant[]
  public towerList: Tower[]
  public streetList: Street[]
  public intersectionList: Array<{beginning: Street, target: Street}>
  public componentUpdateTrigger: any
  public currentTick: number
  public tickSpeed: number
  public spawnThreshold: number
  public continueTimer: boolean
  public experimentType: number
  public decayStrength: number
  public totalDamageDealt: number
  public shortestPathLength: number
  public pheromoneIncreaseStrength: number = 0.01
  public towersEnabled: boolean = true
  public amountOfDeadAnts: number
  public amountOfSpawnedAnts: number
  public substractForDeath: boolean = false
  public pheromoneTarget: number = 0.8
  public targetIsAmountOfAnts: boolean = false
  public maxTicks: number = -1
  constructor(fieldSizeX: number = 10, fieldSizeY: number = 10) {
    this.fieldSizeX = fieldSizeX
    this.fieldSizeY = fieldSizeY
    this.towerList = []
    this.streetList = []
    this.antList = []
    this.intersectionList = []
    this.currentTick = 0
    this.tickSpeed = 1800
    this.spawnThreshold = 3
    this.continueTimer = false
    this.componentUpdateTrigger = {setState: noOp}
    this.experimentType = experimentChoices.onlyOnSuccess
    this.decayStrength = -0.007
    this.totalDamageDealt = 0
    this.shortestPathLength = 99999999999999
    this.amountOfDeadAnts = 0
    this.amountOfSpawnedAnts = 0
  }
  /**
   * Check wether a certain position in the map is already occupied by a tile
   * @param x Position
   * @param y Position
   */
  public isOccupied(x: number, y: number) {
    const fullList: Tile[] = union<Tile>(this.towerList, this.streetList)
    for (const tile of fullList) {
      if (tile.occupies(x, y)) {
        return tile
      }
    }
    return null
  }

  /**
   * Returns true if successfull false otherwise
   * @param newTile Tile to insert
   */
  public addTile(newTile: Tile) {
    if (isNil(this.isOccupied(newTile.xPos, newTile.yPos))) {
      const addTo: any = newTile.tileType === tileTypes.tower ? this.towerList : this.streetList
      addTo.push(newTile)
      return true
    }
    return false
  }
  public getNeighborPositions(xPos: number, yPos: number, range: number) {
    const toCheck: Array<{x: number, y: number}> = []
    for (let i = 1; i <= range; i++) {
      toCheck.push({x: xPos + i, y: yPos})
      toCheck.push({x: xPos - i, y: yPos})
      toCheck.push({x: xPos, y: yPos + i})
      toCheck.push({x: xPos, y: yPos - i})
      toCheck.push({x: xPos + i, y: yPos + i})
      toCheck.push({x: xPos + i, y: yPos - i})
      toCheck.push({x: xPos - i, y: yPos + i})
      toCheck.push({x: xPos - i, y: yPos - i})
    }
    return toCheck
  }
  public getNeighbors: (xPos: number, yPos: number, range: number) => Tile[] = (xPos, yPos, range) => {
    const tiles: Tile[] = []
    // TODO find neighbors
    const toCheck = this.getNeighborPositions(xPos, yPos, range)
    for (const checkMe of toCheck) {
      const found = this.isOccupied(checkMe.x, checkMe.y)
      if (!isNil(found)) {
        tiles.push(found)
      }
    }
    return tiles
  }
  public getNeighborStreets: (xPos: number, yPos: number, range: number) => Street[] = (xPos, yPos, range) => {
    const back: any = filter(this.getNeighbors(xPos, yPos, range), (tile) => tile.tileType === tileTypes.street)
    return back
  }
  public getStart: () => Street = () => {
    return this.streetList[0]
  }
  public getTarget: () => Street = () => {
    const ret: any = find(this.streetList, (a: Street) => a.isTarget())
    return ret
  }
  /**
   * Process ticks for a game
   * at first towers deal damage and kill ants
   * then ants can make their move
   */
  public processTick: () => void = () => {
    for (const tower of this.towerList) {
      tower.makeRoundDamage()
    }
    for (const ant of this.antList) {
      ant.makeNextStep()
    }
    for (const street of this.streetList) {
      street.adjustPheromoneLevel(this.decayStrength)
    }
    this.currentTick++
    if (this.reachedExperimentGoal()) {
      this.stopTimer()
    }
    this.componentUpdateTrigger.setState({tickNum: this.currentTick})
  }
  public spawnNewAnt: (amount: number) => void = (amount = 1) => {
    const start = this.getStart()
    for (let i = 0; i < amount; i++) {
      const ant = new Ant(start)
      start.enter(ant)
      this.antList.push(ant)
    }
    this.amountOfSpawnedAnts += amount
  }

  public removeAnt: (ant: Ant) => void = (ant) => {
    remove(this.antList, (a1) => a1.uniqueId === ant.uniqueId)
  }
  public registerComponentToUpdate = (setStateMethod: any) => {
    this.componentUpdateTrigger = setStateMethod
  }

  public startTimer() {
    this.continueTimer = true
    if (isNil(this.timerId)) {
      console.log("timerid is null starting timer")
      this.timerId = setTimeout(this.processTime, this.tickSpeed)
    }
  }
  public stopTimer() {
    this.continueTimer = false
    if (!isNil(this.timerId)) {
      clearTimeout(this.timerId)
      this.timerId = null
    }
  }
  public processTime = () => {
    this.spawnNewAnt(randomNum(0, this.spawnThreshold))
    this.processTick()
    if (this.continueTimer) {
      setTimeout(this.processTime, this.tickSpeed)
    }
  }
  public reachedExperimentGoal() {
    if (this.maxTicks > 0 && this.currentTick >= this.maxTicks) {
      return true
    }
    if (this.targetIsAmountOfAnts) {
      return this.getTarget().currentVisitors.length >= this.pheromoneTarget
    } else {
      const relevantStreets = filter(this.streetList, ["relevantEvaluationTarget", true])
      let satisifed = false
      for (const street of relevantStreets) {
        if (this.pheromoneTarget <= street.pheromoneLevel) {
          satisifed = true
        }
      }
      return satisifed
    }
  }
}

export default GameField
export { experimentChoices }

\end{lstlisting}


\section{Source Code for Tile-Class}
\label{src:logictile}
\begin{lstlisting}
import { isNil, isObject } from "lodash"
import GameField from "./map"

const tileTypes = {
  street: 3,
  tower: 6
}

class Tile {
  public xPos: number
  public yPos: number
  public tileType: number
  public gameField: GameField
  public constructor(xPos: number, yPos: number, tileType: number, gameField: GameField) {
    this.xPos = xPos
    this.yPos = yPos
    this.tileType = tileType
    this.gameField = gameField
  }
  public occupies: (x: number, y: number) => boolean =
  (x: number, y: number) => {
    return x === this.xPos && y === this.yPos
  }
  public sameTile: (compare?: Tile) => boolean = (compare) => {
    if (isNil(compare) || !isObject(compare)) {
      return false
    }
    return this.xPos === compare.xPos && this.yPos === compare.yPos
  }
}

export default Tile
export { tileTypes }

\end{lstlisting}

\section{Source Code for Street-Class}
\label{src:logicstreet}
\begin{lstlisting}
import { findIndex, remove } from "lodash"
import Ant from "./ant"
import GameField from "./map"
import Tile, { tileTypes } from "./tile"

const direction = {
  up: 2, down: 3, left: 5, right: 6
}

const getDirFromNum: (dir: number) => number[] = (dir) => {
  switch (dir) {
    case direction.up: return [0, -1]
    case direction.down: return [0, 1]
    case direction.left: return [-1, 0]
    case direction.right: return [1, 0]
    default: return [0, 1]
  }
}

class Street extends Tile {
  public next: Street[]
  public previous: Street[]
  public currentVisitors: Ant[]
  public pheromoneLevel: number
  public relevantEvaluationTarget: boolean = false
  constructor(xPos: number, yPos: number, gameField: GameField) {
    super(xPos, yPos, tileTypes.street, gameField)
    this.next = []
    this.previous = []
    this.currentVisitors = []
    this.pheromoneLevel = 0
  }
  public isTarget: () => boolean = () => {
    return this.next.length === 0
  }
  public isStart: () => boolean = () => {
    return this.previous.length === 0
  }
  public multiExtend: (extendToDir: number, amountOfTimes: number) => Street = (extendToDir, amountOfTimes) => {
    let endOfAdding: any = this
    for (let i = 0; i < amountOfTimes; i++) {
      endOfAdding = endOfAdding.extend(extendToDir)
    }
    return endOfAdding
  }
  public extend: (extendToDir: number) => Street = (extendToDir) =>  {
    const directionHelper = getDirFromNum(extendToDir)
    const nextStreet = new Street(
      this.xPos + directionHelper[0],
      this.yPos + directionHelper[1],
      this.gameField
    )
    if (this.gameField.addTile(nextStreet)) {
      this.next.push(nextStreet)
      nextStreet.previous.push(this)
      return nextStreet
    }
    return this
  }
  public addIntersection(intersec: Street) {
    this.next.push(intersec)
    intersec.previous.push(this)
  }
  public enter(ant: Ant) {
    if (findIndex(this.currentVisitors, (e: Ant) => e.uniqueId === ant.uniqueId) !== -1) {
      return false
    }
    return this.currentVisitors.push(ant)
  }
  public leave(ant: Ant) {
    return remove(this.currentVisitors, (e: Ant) => e.uniqueId === ant.uniqueId)
  }
  public adjustPheromoneLevel(by: number) {
    this.pheromoneLevel += by
    if (this.pheromoneLevel < 0) {
      this.pheromoneLevel = 0
    }
    if (this.pheromoneLevel > 1) {
      this.pheromoneLevel = 1
    }
  }
}

export default Street
export { direction }

\end{lstlisting}

\section{Source Code for Tower-Class}
\label{src:logictower}
\begin{lstlisting}

import { remove, sortBy, union } from "lodash"
import randomNum from "../randomNum"
import Ant from "./ant"
import GameField from "./map"
import Tile, { tileTypes } from "./tile"

class Tower extends Tile {
  public damagableAnts: number
  public damageAmount: number
  public towerRange: number
  public towerType: string
  constructor(xPos: number, yPos: number, gameField: GameField) {
    super(xPos, yPos, tileTypes.tower, gameField)
    this.damagableAnts = 7
    this.towerRange = 1
    this.damageAmount = 13
  }
  public makeRoundDamage: () => void = () => {
    // find streets in range and damage ant
    const possibleTargets = this.gameField.getNeighborStreets(this.xPos, this.yPos, this.towerRange)
    let hittableAnts: Ant[] = []
    for (const target of possibleTargets) {
      hittableAnts = union<Ant>(hittableAnts, target.currentVisitors)
    }
    // console.log("hittableAnts are", hittableAnts)
    let antsHit = 0
    while (antsHit < this.damagableAnts && hittableAnts.length > 0) {
      hittableAnts = sortBy(hittableAnts, ["hp", "uniqueId"])
      const ant = hittableAnts[0]
      ant.hurt(this.damageAmount)
      this.gameField.totalDamageDealt += this.damageAmount
      antsHit++
      if (ant.hp <= 0) {
        remove(hittableAnts, (antToCheck) => antToCheck.uniqueId === ant.uniqueId)
      }
    }
  }
}

export default Tower

\end{lstlisting}

\section{Source Code for Ant-Class}
\label{src:logicant}
\begin{lstlisting}
import { each, filter, find, isArray, isEqual, map, union, uniq } from "lodash"
import randomNum from "../randomNum"
import { experimentChoices } from "./map"
import Street from "./street"
import Tile from "./tile"
let currentmaxid = 1

const makeWeightedChoice = (inputArr: Array<{choiceIndex: number, choiceWeight: number}>) => {
  let sumOfWeights = 0
  const validRanges: Array<{min: number, max: number, index: number}> = []
  for (const item of inputArr) {
    validRanges.push({index: item.choiceIndex, min: sumOfWeights, max: sumOfWeights + item.choiceWeight})
    sumOfWeights += item.choiceWeight
  }
  const findInRange = randomNum(0, sumOfWeights)
  const foundRange: any = find(validRanges,
    (toCheck) => findInRange >= toCheck.min && findInRange <= Math.ceil(toCheck.max))
  return foundRange.index
}

const containsStreet = (streetList: Tile[], street: Tile) => {
  let contains = false
  each(streetList,  (inspect) => {
    if (street.sameTile(inspect)) {
      contains = true
      return false
    }
    return true
  })
  return contains
}

class Ant {
  public currentlyOn: Street
  public walkedPath: Street[]
  public uniqueId: number
  public hp: number
  public constructor(currentTile: Street) {
    this.currentlyOn = currentTile
    this.uniqueId = currentmaxid
    this.walkedPath = [currentTile]
    this.hp = 50
    currentmaxid++
  }
  public makeNextStep() {
    if (this.currentlyOn.isTarget()) {
      // Already done
      return
    }
    // also consider going backwards at a crossing by unionizing next and previous and
    // then filtering out the tile that would mean going backwards
    const nextOptions = filter(union(this.currentlyOn.next, this.currentlyOn.previous), (street) => {
      return !this.currentlyOn.sameTile(street) &&
        !street.sameTile(this.walkedPath[Math.max(0, this.walkedPath.length - 1)])
    })
    if (isArray(nextOptions) && nextOptions.length > 0) {
      let i = 0
      const weightedOpts = map(nextOptions, (street) => {
        const obj = {choiceIndex: i, choiceWeight: Math.max(1, street.pheromoneLevel * 100)}
        i++
        return obj
      })
      const nextTileIndex = makeWeightedChoice(weightedOpts)

      const nextTile = nextOptions[nextTileIndex]
      if (this.currentlyOn.gameField.experimentType === experimentChoices.continousVapor) {
        if (!containsStreet(this.walkedPath, this.currentlyOn)) {
          this.currentlyOn.adjustPheromoneLevel(this.currentlyOn.gameField.pheromoneIncreaseStrength)
        }
      }
      this.walkedPath.push(this.currentlyOn)
      this.currentlyOn.leave(this)
      nextTile.enter(this)
      this.currentlyOn = nextTile
      if (nextTile.isTarget()) {
        this.walkedPath.push(nextTile)
        // Use uniq here so each crossed street tile only gets evaluated once
        const streetsToEvaluate = uniq(this.walkedPath)
        if (nextTile.gameField.experimentType === experimentChoices.onlyOnSuccess) {
          for (const targetCrossed of this.walkedPath) {
            targetCrossed.adjustPheromoneLevel(nextTile.gameField.pheromoneIncreaseStrength)
          }
        } else if (nextTile.gameField.experimentType === experimentChoices.shortestPathOnly) {
          if (nextTile.gameField.shortestPathLength >= this.walkedPath.length) {
            nextTile.gameField.shortestPathLength = this.walkedPath.length
            for (const street of this.walkedPath) {
              street.adjustPheromoneLevel(nextTile.gameField.pheromoneIncreaseStrength)
            }
          }
        } else if (nextTile.gameField.experimentType === experimentChoices.shortestPathWeight) {
          if (nextTile.gameField.shortestPathLength >= this.walkedPath.length) {
            nextTile.gameField.shortestPathLength = this.walkedPath.length
          }
          const modifier =  nextTile.gameField.shortestPathLength / this.walkedPath.length
          for (const street of this.walkedPath) {
            street.adjustPheromoneLevel(nextTile.gameField.pheromoneIncreaseStrength * modifier)
          }
        }
      }
    }
  }
  public hurt(damageDealt: number) {
    this.hp -= damageDealt
    if (this.hp <= 0) {
      this.currentlyOn.gameField.removeAnt(this)
      this.currentlyOn.gameField.amountOfDeadAnts++
      this.currentlyOn.leave(this)
      if (this.currentlyOn.gameField.towersEnabled && this.currentlyOn.gameField.substractForDeath) {
        for (const targetCrossed of this.walkedPath) {
          targetCrossed.adjustPheromoneLevel(this.currentlyOn.gameField.decayStrength)
        }
      }
    }
  }
}

export default Ant

\end{lstlisting}

\section{Source Code for Ant-Displayment}
\label{src:uiant}
\begin{lstlisting}
import { findIndex } from "lodash"
import * as React from "react"
import { VelocityComponent } from "velocity-react"
import Ant from "../logic/ant"
import Map from "../logic/map"
import { getPositionForTile } from "./positionUtil"

class TowerDisplay extends React.Component<{
  ant: Ant
  gameField: Map
}, any> {
  public render() {
    const classes = "ant"
    const antStyle: any = getPositionForTile(this.props.ant.currentlyOn,
      this.props.gameField, 10)
    antStyle.opacity = this.props.ant.hp <= 0 || this.props.ant.currentlyOn.isStart() ||
      this.props.ant.currentlyOn.isTarget() ? 0 : 1
    return (
      <VelocityComponent
        animation={antStyle}
        duration={this.props.gameField.tickSpeed * 1.3}
      >
        <div className={classes} />
      </VelocityComponent>
    )
  }
}

export default TowerDisplay

\end{lstlisting}

\section{Source Code for Game-Simulation Test}
\label{src:testgamesim}
\begin{lstlisting}
import * as fs from "fs"
import { map, merge } from "lodash"
import { mkdir } from "shelljs"
import { experimentChoices } from "../logic/map"
import { createMap as mirroredwithtower } from "../maps/experiment/mirroredWithTower"
import { createMap } from "../maps/experiment/shortAndLong"
import { createMap as shortTowersAndLongMap } from "../maps/experiment/shortTowersAndLong"
import { createMap as squareMazeMap } from "../maps/experiment/squareMaze"
import { GamefieldCreator } from "./experiment"
import makePlot from "./makePlot"
import makePlotScatter, { iterationTargets } from "./makePlotScatter"
import tryoutAllSettings from "./tryoutAllSettings"

const start = Date.now()

const makeExperiment = async (name: string, mapCreator: GamefieldCreator,
                              testWithAndWithoutDeath: boolean, settings: any) => {
const res = await tryoutAllSettings(mapCreator, settings, 50)
mkdir("results")
fs.writeFileSync(`results/${name}.json`, JSON.stringify(res, null, 2))
await makePlot(name, res, testWithAndWithoutDeath)
}

const makeExperimentScatter =
async (name: string, mapCreator: GamefieldCreator, testWithAndWithoutDeath: boolean, settings: any) => {
  const fullRes: any[] = []
  for (const setting of settings) {
    const settingsToUse = map(
      iterationTargets, (pheromoneTarget: any) => {
      return merge({}, setting, {pheromoneTarget, maxTicks: pheromoneTarget * 10})
    })
    console.log("got settings", settingsToUse)
    const res = await tryoutAllSettings(mapCreator, settingsToUse, 10, false)
    for (const a of res) {
      fullRes.push(a)
    }
  }
  mkdir("results")
  fs.writeFileSync(`results/${name}-scatter.json`, JSON.stringify(fullRes, null, 2))
  await makePlotScatter(name, fullRes, testWithAndWithoutDeath)
  const res = await tryoutAllSettings(mapCreator, settings, 30)
  fs.writeFileSync(`results/${name}.json`, JSON.stringify(res, null, 2))
  await makePlot(name, res, testWithAndWithoutDeath)
}

const varyWithAllExperimentTypes = (baseSettings: any, nameAddon: string = "") => {
  return [
    merge({
      name: nameAddon + "Continous Vapor",
      experimentType: experimentChoices.continousVapor
    }, baseSettings),
    merge({
      name: nameAddon + "Success only",
      experimentType: experimentChoices.onlyOnSuccess
    }, baseSettings),
    merge({
      name: nameAddon + "SP Only",
      experimentType: experimentChoices.shortestPathOnly
    }, baseSettings),
    merge({
      name: nameAddon + "SP Weight",
      experimentType: experimentChoices.shortestPathWeight
    }, baseSettings)
  ]
}

const runSettingOnAllMaps = async (settings: any, name: string, testWithAndWithoutDeath: boolean) => {
  const experimentStart = Date.now()
  await makeExperimentScatter(name + "shortandlong", createMap, testWithAndWithoutDeath,
  varyWithAllExperimentTypes(settings))
  await makeExperimentScatter(name + "shortandlongwithtowers",
    shortTowersAndLongMap, testWithAndWithoutDeath, varyWithAllExperimentTypes(settings))
  await makeExperimentScatter(name + "mirroredwithtower",
    mirroredwithtower, testWithAndWithoutDeath, varyWithAllExperimentTypes(settings))
  await makeExperimentScatter(name + "squaremaze",
    squareMazeMap, testWithAndWithoutDeath, varyWithAllExperimentTypes(settings))
  console.log("finished running experiment after " + (Date.now() - experimentStart) + " ms")
  console.log("current total runtime " + (Date.now() - start) + " ms")
}

const runAllExperiments = async () => {
  await runSettingOnAllMaps({
    maxTicks: 9999,
    pheromoneTarget: 400,
    targetIsAmountOfAnts: true,
    decayStrength: -0.007,
    spawnThreshold: 14
  }, "normal", true)

  await runSettingOnAllMaps({
    maxTicks: 9999,
    pheromoneTarget: 400,
    targetIsAmountOfAnts: true,
    decayStrength: -0.007,
    spawnThreshold: 30
  }, "high spawn normal decay", true)

  await runSettingOnAllMaps({
    maxTicks: 9999,
    pheromoneTarget: 400,
    targetIsAmountOfAnts: true,
    decayStrength: -0.02,
    spawnThreshold: 30
  }, "high spawn high decay", true)

  await runSettingOnAllMaps({
    maxTicks: 9999,
    pheromoneTarget: 400,
    targetIsAmountOfAnts: true,
    decayStrength: -0.02,
    spawnThreshold: 4
  }, "low spawn high decay", true)

  await runSettingOnAllMaps({
    maxTicks: 9999,
    pheromoneTarget: 400,
    targetIsAmountOfAnts: true,
    decayStrength: -0.0001,
    spawnThreshold: 4
  }, "low spawn low decay", true)

  await runSettingOnAllMaps({
    maxTicks: 9999,
    pheromoneTarget: 400,
    targetIsAmountOfAnts: true,
    decayStrength: -0,
    spawnThreshold: 4
  }, "low spawn no decay", true)

  console.log("total runtime " + (Date.now() - start) + " ms")
  return true
}

const makestuff = async () => {
  const baseSettings = {
    maxTicks: 9999,
    pheromoneTarget: 400,
    targetIsAmountOfAnts: true,
    decayStrength: -0.007,
    spawnThreshold: 14
  }
  await makeExperimentScatter("mirroredwithtower", mirroredwithtower, true,
  [
    merge({
      name: "Normal",
      experimentType: experimentChoices.shortestPathWeight
    }, baseSettings),
    merge({
      name: "High Decay",
      experimentType: experimentChoices.shortestPathWeight
    }, baseSettings, {
      decayStrength: -0.02
    }),
    merge({
      name: "Low Decay",
      experimentType: experimentChoices.shortestPathWeight
    }, baseSettings, {
      decayStrength: -0.0007
    }),
    merge({
      name: "High Increase",
      experimentType: experimentChoices.shortestPathWeight
    }, baseSettings, {
      pheromoneIncreaseStrength: 0.02
    }),
    merge({
      name: "Low Increase",
      experimentType: experimentChoices.shortestPathWeight
    }, baseSettings, {
      pheromoneIncreaseStrength: 0.0007
    }),
    merge({
      name: "High Increase & Decay",
      experimentType: experimentChoices.shortestPathWeight
    }, baseSettings, {
      pheromoneIncreaseStrength: 0.03,
      decayStrength: -0.02
    }),
    merge({
      name: "Low Increase & Decay",
      experimentType: experimentChoices.shortestPathWeight
    }, baseSettings, {
      pheromoneIncreaseStrength: 0.001,
      decayStrength: -0.0007
    })
  ])
}

export default makestuff

\end{lstlisting}



%
% The following two commands are all you need in the
% initial runs of your .tex file to
% produce the bibliography for the citations in your paper.
%\bibliographystyle{abbrv}
%\bibliography{main}  % sigproc.bib is the name of the Bibliography in this case
% You must have a proper ".bib" file
%  and remember to run:
% latex bibtex latex latex
% to resolve all references
%
% ACM needs 'a single self-contained file'!
%
%APPENDICES are optional
%\balancecolumns
\bibliographystyle{abbrv}
\bibliography{report} 

\end{document}
