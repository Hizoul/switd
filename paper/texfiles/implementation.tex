This chapter is split up into two subparts: Tech-Stack and Actual Implementation. The Tech-Stack findable in Section \ref{sec:implementationtech}, explains the libraries and techonolgies that enable our implementation. The Actual Implementation in Section \ref{sec:implementationactual} covers how we ensured a minimally working tower-defense game.

\subsection{Technical Stack}
\label{sec:implementationtech}
For our expirment to work, we require a programatically executable and influencable version of a tower defense game. There are definitely many open-source solutions out there\footnote{A search on github for "tower defense" in javascript returned 380 results.\cite{githubtowerdef}}, but for our ant colony optimization applicability we only require a very basic and simple implementation, so we decided to quickly roll out our own. How this actual Implementation looks is laid out in Section \ref{sec:implementationactual}

For the implementation we chose to create a web-tool through the usage of Javascript as a programming langauge. It provides the major advantage that, if we want to let someone else run our program to verify our conclusions, they can simply visit a link in their browser, be it mobile, desktop or gaming-console, to interact with our tool and visually see ant's deciding their ways and the pheromone-levels on the street slowly increasing\cite{curran2012future}.

We will be using typescript\cite{libstypescript} as an intermediate language to Javacsript because it supports react's JSX-Syntax per default so no need to use a transpiler like babel, and having type-suggestions etc. increases developer workflow speed as well as eases understanding of code for other team-members.

In order to make sure changes do not break any existing behaviour we use the test runner called \textit{jest}\cite{libsjest}. It comes with the handy utility  \textit{matchSnapshot}\footnote{Example Usage see Section \ref{src:testgamesim}}, which allows us to match the whole current gamestate against a snapshot and immediately see unwanted changes if they occur. This \textit{matchSnapshot}-Function is also capable of capturing a snapshot of our entire UI-render, which means we are able to achieve 100\%-Line-Code-Coverage

In order to be able to easily implement the UI and add Animations merely based on the state of the game we chose to opt for \textit{react}\cite{libsreact}, with the easy \textit{velocity-react}\cite{libsreactvelocity} animation extension. With this we only have to write the position that we want for example an ant to be displayed at for the current tick, and \textit{velocity-react} remembers the previous values and makes sure that the changes in position values are then animated. This creates the illusion of the ants walking along the path as the ticks progress\footnote{Actual Code see Section \ref{src:uiant}}. This also inherently comes with some weird animation behaviour, for example when setting the tickspeed to high then the ants position animation is merely from start to finish so it may appear to not walk on a valid street while in the backend it definitely did. Also with each change in the tick there is a slight "freeze" in the movement of the ants.

\subsection{Actual Implementation}
\label{sec:implementationactual}

In order to Implement a tower defence game one has to identify the key-objects that play different roles in the game namely that is
\begin{itemize}
\item The walkable path with special start and target point
\item Ants walking along this path
\item Towers trying to damage these Ants
\end{itemize}
First of we store our whole map-state as well as the map-layout itself in a single class called \textit{GameField}\footnote{Source-Code in Section \ref{src:logicgame}}. The game should hold a simple 2 dimensional coordinate system, that has an x and y (width and height) constraint, so we can place our key-objects within the Field. Towers\footnote{Source-Code in Section \ref{src:logictower}} and Street\footnote{Source-Code in Section \ref{src:logicstreet}}-Parts both originate from the class \textit{Tile}\footnote{Source-Code in Section \ref{src:logictile}} which lets us position them within the coordinate system of the \textit{GameField} and makes sure they cannot overlap.
The Streets are a simple adjacency List meaning each street-part only nows its previous and next nodes and nothing else. This makes our street-network a simple network suitable for ACO. Streets also have a list of ants that are currently on them and the current numerical pheromone level we introduced in Section \ref{sec:introduction}. This is already all the data needed for the \textit{Ant}\footnote{Source-Code in Section \ref{src:logicant}}-Class to make it's ACO-Based Routing decisions!
The Tower is also very simple it has a range, damage amount per hit and amount of hits per tick. Each Tick it will look for all streets in range and then attack as many ants as possible. It will try to damage ants that are the furthest and have the fewest HP first.

Bringing all this domain-logic to a visually interpretable is also pretty simple with the given data-structure. For each Tile we simply display a basic square via an HTML-\textit{div}-Element. Since we have a certain x and y-limit for our GameField mentioned earlier, we can calculate the exact width, height and position of each tile width within our window-JS-Object\cite[P. 570]{goodman2002dynamic}.

Each Game Instance also holds a variable that determines what algorithm the Ant's use to determine which node to walk to next, and how strongly pheromones are influenced and how they decay. Through that we will be able to run different kinds of experiments on different maps.

The Game itself is Tick-Based. In each tick we first let the tower's deal all their damage. Then all the ants decide where they walk to next. After that has happened the general decay of pheromones is processed if wanted.

This is already all the implementation. To sum it all up: We have an isolated GameField-Class that can instantiate a whole test-setup including map, ants and tick-progression, so in theory it is possible to use this code to run multiple expirements at the same time.